% !TeX root = ../main.tex

\section{Introduction}

Anime has grown from a niche subculture into a global phenomenon, influencing
millions of viewers and shaping media trends across countries. Beyond entertainment,
anime data offers a fascinating window into human preferences, cultural diffusion,
and social behavior. Analyzing how users rate and interact with anime provides opportunities
to understand patterns in collective taste, popularity dynamics, and even cross-cultural
differences in media consumption.

Moreover, anime can be viewed not only as a collection of individual titles, but as a complex
social network. Genres, studios, franchises, and shared audiences form clusters and hubs:
tightly knit communities built around specific themes, styles, or narrative structures.
This creates a rich environment for studying community formation, network effects in popularity,
and the emergence of subcultures. In this sense, anime serves as a microcosm of larger social
processes — how groups organize, how trends spread, and how cultural identities form and evolve.

From a data-scientific perspective, anime datasets present multiple challenges and opportunities,
offering a rare combination of numerical, categorical, and textual information embedded in a naturally
occurring, socially meaningful structure. This makes them well-suited for analyzing real patterns in
human behavior through feature engineering and exploratory data analysis. They allow us to explore
questions such as how different genres appeal to different audiences, how collective preferences
change over time, how collective user behavior emerges from individual ratings, and how user
communities structure themselves around shared interests.

In today's environment, where online platforms dominate media consumption, understanding patterns
in user ratings and content characteristics has both practical and scientific value. Insights derived
from anime data can inform recommendation systems, guide content creation, and serve as a microcosm
for analyzing trends in larger entertainment ecosystems. Studying anime data helps illuminate the
mechanisms of popularity, personalization, and cultural diffusion, providing insights that generalize
far beyond this specific medium. In other words, anime is not only culturally significant, but also
a compelling, real-world playground for developing and demonstrating data science methods.

For this project, we use the \textit{Anime Recommendation Database 2020}
from Kaggle\cite{hernan4444anime2020} \footnote{\url{https://www.kaggle.com/datasets/hernan4444/anime-recommendation-database-2020}},
which contains detailed information from \textbf{MyAnimeList} about over 16,000 anime titles and ratings
from more than 300,000 users. This dataset is rich, heterogeneous, and somewhat messy — exactly the kind
of data that makes data science exciting. 

The key questions we aim to explore include \cite{hernan4444anime2020}:

\begin{itemize}
    \item Which anime are the most highly rated, and which are underrated compared to their popularity?
    \item How do user ratings vary across genres, release years, and demographics?
    \item Can we identify clusters of similar anime based on genre, themes, and user ratings?
    \item Are there patterns in user behavior, such as tendencies to rate certain genres more favorably?
\end{itemize}

Through exploratory data analysis, we hope to extract meaningful insights, for example:

\begin{itemize}
    \item Genre trends: which genres consistently attract high ratings.
    \item Temporal patterns: how ratings and popularity evolve over time.
    \item Relationships between anime features (genre, episodes, year) and user ratings.
\end{itemize}

However, the dataset also has inherent limitations. We cannot reliably infer causal relationships from these
ratings, nor can we accurately predict the future success of an anime solely based on historical user ratings.
Additionally, user bias, incomplete data, and differences in rating scales limit the conclusions we can draw.

By focusing on data cleaning, exploratory analysis, and interpretation, this project demonstrates how raw,
real-world data can be transformed into actionable insights, while also highlighting the boundaries of what
such data can tell us.

The analysis pipeline and code are provided in the companion notebook \cite{animeprojectnotebook2025}.