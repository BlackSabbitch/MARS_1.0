% !TeX root = main.tex

\section{Dataset Overview}

MyAnimeList (MAL) is one of the largest online platforms dedicated to anime and manga tracking. 
Users maintain personal lists of watched titles, assign numerical ratings, write reviews, and participate in community discussions. 

Although MAL does not represent the entire global audience, it functions as a \emph{large-scale, organically formed social network} structured around media consumption. Its importance for data analysis stems from several factors:
\begin{itemize}
    \item it has millions of active users with voluntarily provided preference data;
    \item users form clusters and communities around genres, studios, eras, and specific titles;
    \item the rating patterns reflect collective taste dynamics, hype cycles, long-tail phenomena, and cross-cultural differences;
    \item the anime graph (users $\times$ titles) behaves like a sparse bipartite network with hubs, where highly popular titles act as central connectivity points.
\end{itemize}

For these reasons, MAL is a valuable source for studying recommendation systems, preference modeling, popularity prediction, and structural patterns of entertainment consumption.

\subsection{Dataset Structure}

We use the 2020 ``Anime Recommendation Database'' from Kaggle, which combines several processed dumps of MyAnimeList:
\footnote{Where external statistics are used (industry reports, demographic data), and where it is possible, we use sources closest
to 2020 to maintain temporal consistency.}
\paragraph{anime.csv} Metadata for approximately 17,000 titles, with columns including:
\begin{itemize}
    \item identifiers: \texttt{MAL\_ID}, \texttt{Name}, \texttt{English name}, \texttt{Japanese name};
    \item quality and popularity metrics: \texttt{Score}, \texttt{Ranked}, \texttt{Popularity}, \texttt{Members}, \texttt{Favorites};
    \item categorical descriptors: \texttt{Genres}, \texttt{Type}, \texttt{Source}, \texttt{Studios}, \texttt{Producers};
    \item structural info: \texttt{Episodes}, \texttt{Duration}, \texttt{Rating} (age restriction);
    \item engagement counters: \texttt{Watching}, \texttt{Completed}, \texttt{Dropped}, etc.;
    \item per-score vote counts: \texttt{Score-1} $\dots$ \texttt{Score-10}.
\end{itemize}

\paragraph{anime\_with\_synopsis.csv} A reduced version containing \texttt{MAL\_ID}, \texttt{Name}, \texttt{Score}, \texttt{Genres}, \texttt{Synopsis}. Useful for NLP tasks such as clustering by textual description.

\paragraph{animelist.csv} Contains approximately 300 million user–anime interactions, with columns: \texttt{user\_id}, \texttt{anime\_id}, \texttt{rating}, \texttt{watching\_status}, \texttt{watched\_episodes}. This is the core behavioral dataset representing the user–item matrix.

\paragraph{rating\_complete.csv} A filtered version of \texttt{animelist.csv} containing only rows with \texttt{watching\_status = 2} (``Completed''), columns: \texttt{anime\_id}, \texttt{user\_id}, \texttt{rating}. Commonly used for training recommender systems.

\paragraph{watching\_status.csv} Lookup table pairing each integer code with a textual description (``Currently Watching'', ``Completed'', etc.).

\subsection{Strengths of the Dataset}

Despite being collected from an entertainment platform, the dataset has several significant advantages for data science practice:

\begin{itemize}
    \item \textbf{Large scale:} Tens of millions of ratings across thousands of titles enable analysis of long-tail
    distributions, user segmentation, genre-level statistics, and popularity dynamics.

    \item \textbf{Natural heterogeneity of users:} No incentive to game the system; tastes are diverse and clusters form organically.

    \item \textbf{Multiple complementary tables:} Metadata, textual features, and behavioral interactions allow
    content-based, collaborative, graph-based, and hybrid recommender analyses.

    \item \textbf{Excellent for methodological demonstration:} Useful for data cleaning, exploratory data analysis
    (EDA), imputation, outlier detection, recommendation algorithms, and metadata fusion with NLP.

\end{itemize}

\subsection{Limitations and Potential Biases}

The dataset reflects behaviors of a specific community and inherits biases from the platform:

\begin{itemize}
    \item \textbf{Geographical bias:} MAL is most popular in North America, parts of Europe, and Southeast Asia. 
    For example, China is one of the global leaders in anime licensing \cite{animeindustry2020}, but Chinese
    users are absent on MAL.
    
    \item \textbf{Cultural and language bias:} English-speaking communities dominate the dataset.

    \item \textbf{Sparse user--item matrix:} Most users have watched only a small fraction of all anime; 
    this affects collaborative filtering model performance and cold-start dynamics.

    \item \textbf{Artifacts and inconsistency in dataset:} Example for \texttt{rating\_complete.csv}:
    filtering by \texttt{watching\_status = Completed} is not sufficient. Some users marked anime as completed
    but watched fewer episodes than the total and gave non-zero ratings. Since MAL does not allow rating = 0,
    zero ratings indicate \emph{no vote} rather than dislike. These cases are rare (\textasciitilde0.1\%)
    but must be considered during cleaning.

    \item \textbf{No timestamps:} Limits temporal modeling of tastes and popularity.

    \item \textbf{Caution in interpreting popularity:} The dataset reflects preferences of \emph{dedicated fans}
    rather than the general population. So, MAL ratings and engagement metrics do not directly translate to global popularity.
    MAL users are biased towards committed anime fans; casual viewers are underrepresented.

    \item \textbf{Licensing bias:} In many regions, anime consumption occurs primarily through piracy,
    so official licensing or platform metrics severely underestimate actual viewership\footnote{Even in the USA,
    the leader in licensed anime, nearly half of users watch primarily via unofficial services \cite{animeresearch2020}}.

    \item \textbf{MAL - only one of the biggest anime hubs:} Country-level popularity is not reliably inferable
    without additional corrections or external data sources; such adjustments are beyond the scope of this project.

    \end{itemize}
\end{itemize}