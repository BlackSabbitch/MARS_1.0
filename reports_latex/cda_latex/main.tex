% !TeX root = main.tex

% \documentclass{article}
\documentclass[sigconf]{acmart}
\usepackage{graphicx}
\usepackage{amsmath, amssymb}
\usepackage{subcaption}
% --- Метаданные (убираем копирайт для черновика) ---
\settopmatter{printacmref=false} % Убирает блок ссылок ACM внизу
\setcopyright{none}
\renewcommand\footnotetextcopyrightpermission[1]{} % Убирает копирайт
\pagestyle{plain} % Включает нумерацию страниц (для черновика полезно)
%\usepackage[section]{placeins}
\usepackage{float}
\pagestyle{plain}
\settopmatter{printfolios=true, printacmref=false}
\acmConference[Project on Complex Data Analysis]{Project on Complex Data Analysis}{10/12/2025}{Lisbon, Portugal}
\acmYear{2025}
\begin{document}
\title{Evolution of Anime Co-Viewership Networks (2006–2018)}
% 2. АВТОРЫ (Вставляются ПЕРЕД \maketitle)
% Автор 1
\author{Iaroslav Sagan (66661)}
\affiliation{%
  \institution{University of Lisbon}
  \city{Lisbon}
  \country{Portugal}
}
\email{yaroslav_sagan@gmail.com}

% Автор 2
\author{Anna Maksymchuk (66662)}
\affiliation{%
  \institution{University of Lisbon}
  \city{Lisbon}
  \country{Portugal}
}
\email{anna.i.maksymchuk@gmail.com}

\author{Maria Samosudova (66663)}
\affiliation{%
  \institution{University of Lisbon}
  \city{Lisbon}
  \country{Portugal}
}
\email{samosudova@gmail.com}

\begin{abstract}
This study analyzes the structural evolution of the MyAnimeList platform (2006–2018). Utilizing time-sliced networks,
we examine the dynamics of taste communities and user navigation. We employ the Leiden algorithm to trace the genealogy
of clusters from a cohesive core to a fragmented landscape. Regarding user modeling, we find that stochastic random walker
agents diverge from empirical trajectories, indicating the limitations of purely topological simulation. In contrast, machine
learning experiments successfully predict user migration patterns. Crucially, we demonstrate that the inclusion of community-based
clustering metrics significantly improves model accuracy, highlighting the predictive value of mesoscale network features.

\keywords{Social Network Analysis, Community Detection, Dynamic Graphs, Random Walks, User Modeling}
\end{abstract}

\maketitle

\begin{abstract}

This study analyzes the structural evolution of the MyAnimeList platform (2006–2018). Utilizing time-sliced networks,
we examine the dynamics of taste communities and user navigation. We employ the Leiden algorithm to trace the genealogy
of clusters from a cohesive core to a fragmented landscape. Regarding user modeling, we find that stochastic random walker
agents diverge from empirical trajectories, indicating the limitations of purely topological simulation. In contrast, machine
learning experiments successfully predict user migration patterns. Crucially, we demonstrate that the inclusion of community-based
clustering metrics significantly improves model accuracy, highlighting the predictive value of mesoscale network features.

\textbf{Keywords:} Social Network Analysis, Community Detection, Dynamic Graphs, Random Walks, User Modeling.
\end{abstract}

% !TeX root = ../main.tex

\section{Introduction}

Anime has grown from a niche subculture into a global phenomenon, influencing
millions of viewers and shaping media trends across countries. Beyond entertainment,
anime data offers a fascinating window into human preferences, cultural diffusion,
and social behavior. Analyzing how users rate and interact with anime provides opportunities
to understand patterns in collective taste, popularity dynamics, and even cross-cultural
differences in media consumption.

Moreover, anime can be viewed not only as a collection of individual titles, but as a complex
social network. Genres, studios, franchises, and shared audiences form clusters and hubs:
tightly knit communities built around specific themes, styles, or narrative structures.
This creates a rich environment for studying community formation, network effects in popularity,
and the emergence of subcultures. In this sense, anime serves as a microcosm of larger social
processes — how groups organize, how trends spread, and how cultural identities form and evolve.

From a data-scientific perspective, anime datasets present multiple challenges and opportunities,
offering a rare combination of numerical, categorical, and textual information embedded in a naturally
occurring, socially meaningful structure. This makes them well-suited for analyzing real patterns in
human behavior through feature engineering and exploratory data analysis. They allow us to explore
questions such as how different genres appeal to different audiences, how collective preferences
change over time, how collective user behavior emerges from individual ratings, and how user
communities structure themselves around shared interests.

In today's environment, where online platforms dominate media consumption, understanding patterns
in user ratings and content characteristics has both practical and scientific value. Insights derived
from anime data can inform recommendation systems, guide content creation, and serve as a microcosm
for analyzing trends in larger entertainment ecosystems. Studying anime data helps illuminate the
mechanisms of popularity, personalization, and cultural diffusion, providing insights that generalize
far beyond this specific medium. In other words, anime is not only culturally significant, but also
a compelling, real-world playground for developing and demonstrating data science methods.

For this project, we use the \textit{Anime Recommendation Database 2020}
from Kaggle\cite{hernan4444anime2020} \footnote{\url{https://www.kaggle.com/datasets/hernan4444/anime-recommendation-database-2020}},
which contains detailed information from \textbf{MyAnimeList} about over 16,000 anime titles and ratings
from more than 300,000 users. This dataset is rich, heterogeneous, and somewhat messy — exactly the kind
of data that makes data science exciting. 

The key questions we aim to explore include \cite{hernan4444anime2020}:

\begin{itemize}
    \item Which anime are the most highly rated, and which are underrated compared to their popularity?
    \item How do user ratings vary across genres, release years, and demographics?
    \item Can we identify clusters of similar anime based on genre, themes, and user ratings?
    \item Are there patterns in user behavior, such as tendencies to rate certain genres more favorably?
\end{itemize}

Through exploratory data analysis, we hope to extract meaningful insights, for example:

\begin{itemize}
    \item Genre trends: which genres consistently attract high ratings.
    \item Temporal patterns: how ratings and popularity evolve over time.
    \item Relationships between anime features (genre, episodes, year) and user ratings.
\end{itemize}

However, the dataset also has inherent limitations. We cannot reliably infer causal relationships from these
ratings, nor can we accurately predict the future success of an anime solely based on historical user ratings.
Additionally, user bias, incomplete data, and differences in rating scales limit the conclusions we can draw.

By focusing on data cleaning, exploratory analysis, and interpretation, this project demonstrates how raw,
real-world data can be transformed into actionable insights, while also highlighting the boundaries of what
such data can tell us.

The analysis pipeline and code are provided in the companion notebook \cite{animeprojectnotebook2025}.
% !TeX root = main.tex

\section{Data and Methodology}

\subsection{Dataset Acquisition and Preprocessing}

The data originates from a publicly available kaggle datasets aggregated from MAL profiles \cite{hernan4444, azathoth42},
covering the period from 2006 to 2018. Given the platform's predominantly international user base, the dataset reflects global
consumption patterns rather than domestic Japanese trends.

To ensure data quality, we applied a multi-stage filtering pipeline:
\begin{itemize}
    \item \textbf{Bot Detection:} Removal of inactive accounts and profiles exhibiting automated behavior.
    \item \textbf{Percentile-based Truncation:} We excluded users falling into the extreme tails of the activity distribution.
    This removes users with too few votes (insufficient signal for clustering) and those with implausibly high vote counts,
    ensuring the analysis focuses on human-scale consumption patterns.
\end{itemize}

The final processed dataset comprises approximately \textbf{85,000 unique users} and \textbf{6,500 anime titles}.

\subsection{Graph Projection and Topology Construction}
We model the system as a bipartite graph which is subsequently projected into two distinct monopartite networks. The detailed
construction pipelines are described in Sections \ref{sec:pipeline_anime} and \ref{sec:pipeline_users}, respectively.

\subsubsection{Anime-Anime Network}
In this projection, an edge exists if two titles share a common voter. To account for varying audience sizes, we utilized the
\textbf{Jaccard Similarity} \cite{jaccard} index as the edge weight. 
$$ J(A,B) = \frac{|U_A \cap U_B|}{|U_A \cup U_B|} $$
Given the extreme density of the raw projection (where a single popular anime could fully connect thousands of users),
we applied a hard threshold of $J > 0.05$. This effectively prunes weak links formed by random coincidences while preserving
significant genre or fandom connections.

\subsubsection{User-User Network}
Here, an edge connects two users if they have rated the same anime. The edge weight is defined as the raw count of shared titles
(co-votes). A major challenge in this projection is the variance in edge weights, which range from negligible values (2-3 shared
items) to tens of thousands ($10^4$). To address this, we implemented a cutoff threshold: edges were retained only if users shared
more than \textbf{3 titles}.

Anyway, even after thresholding, the user-user raw network projection suffered from extreme density saturation. Popular
"blockbuster" titles (e.g., \textit{Death Note}, \textit{Attack On Titan}) act as super-hubs; a single vote for such a title
effectively connects a user to thousands of others, creating a near-clique structure that obscures genuine taste communities. So,
this titles had to be deleted from the dataset prior to projection.

\subsubsection{Further Sparsification Attempts}
More aggressive sparsification techniques (e.g., Backbone extraction, k-NN) were tested but did not reveal significantly distinct
structural patterns. Consequently, we retained the simpler approach to avoid unnecessary information loss while maintaining
structural clarity.

\subsection{Resulting Topology}
These thresholding strategies proved effective in mitigating the "hairball" phenomenon common in social graphs. The resulting networks
exhibited a graph density in the range of $0.2 - 0.3$, striking a balance between sparsity (for efficient clustering) and connectivity
(preserving the Giant Connected Component).

Only after these topological corrections is the graph subjected to the Leiden community detection algorithm and Random Walk simulations.

\subsection{Computational Framework and Reproducibility}
To ensure reproducibility and handle large-scale temporal networks efficiently, we developed a dedicated modular Python framework
\texttt{project\_cda} \cite{mars_project_repo}, which is a core part of the open-source repository MARS\_1.0. The framework is
designed with a modular architecture to support the full research lifecycle:

\paragraph{Graph Construction Engine}
The graph construction modules \texttt{AnimeGraphBuilder} and \texttt{UserGraphBuilder} modules utilize streaming JSON parsers
(`ijson`) to process massive user interaction logs with minimal memory footprint. They implement the projection logic described
in Section \ref{sec:anime_topology} and support vectorized graph operations via the `igraph` C-core \cite{igraph}, ensuring high performance even
for dense snapshots.

\paragraph{Simulation and Analysis Modules}
The framework includes specialized components for dynamic analysis:
\begin{itemize}
    \item \textbf{CommunityTracker:} Implements a greedy matching algorithm based on Jaccard similarity to trace cluster
    lineage across time steps.
    \item \textbf{RandomCrowd:} An agent-based simulation engine that deploys stochastic walkers to probe network topology
    and user navigation patterns.
    \item \textbf{ClusterEvaluation:} Encapsulates the calculation of structural (Modularity), semantic (Purity), and
    information-theoretic (Entropy) metrics.
\end{itemize}

\paragraph{Data Management and Caching}
Given the computational cost of generating 13 annual snapshots with varying hyperparameters, we implemented a robust caching
system managed by a \texttt{PathManager}. This module enforces idempotency: each experimental configuration (edge weights,
sparsification, clustering algorithm) generates a unique hash signature. If a serialized artifact exists for a given configuration,
it is loaded instantly ("lazy evaluation"), preventing redundant computations. The exact data schema required to run the pipeline
is documented in the repository's \texttt{README.md}.
% !TeX root = main.tex

\section{Topological Evolution of Projected Networks}

In this section, we analyze the structural evolution of the projections derived from the bipartite graph. We first examine the Anime-Anime
network to understand how content relationships shifted over time, followed by an analysis of the User-User network.

% --- ПОДРАЗДЕЛ 3.1: АНИМЕ ---
\subsection{Anime-Anime Network}
\label{sec:anime_topology}

The projected anime-anime network experienced explosive growth over the analyzed period (2006–2018), transforming from a compact,
niche community into a sprawling, heterogeneous ecosystem. This transformation is defined by three primary phenomena: the
densification-sparsification paradox, increasing taste divergence, and the crystallization of a "rich-club" core.

\subsubsection{The Densification-Sparsification Paradox}
The network underwent a dramatic scale expansion: the number of nodes (anime titles) increased from 732 in 2006 to 6,129 in 2018,
while the volume of connections (edges) surged from $\sim$64,000 to $\sim$819,000. However, this volumetric growth reveals a fundamental
structural shift.

While the absolute number of connections increased by an order of magnitude, the potential number of connections grew quadratically ($N^2$).
Consequently, the global Graph Density declined precipitously from 0.2387 (2006) to 0.0436 (2018).

This indicates a transition from a "Village" topology—where the community is small enough for high interconnectedness—to a "Metropolis"
structure. In the modern era, the ecosystem has become highly specialized; while the total volume of interactions is higher, individual
anime titles connect to a significantly smaller fraction of the total population. The network has shifted from a monolithic block to a
spread-out, sparse landscape.

\subsubsection{Increasing Social Distance and Taste Divergence}
To quantify the "cost" of traversing this expanding network, we analyzed weighted path metrics. Since edge weights represent similarity
(Jaccard), the weighted distance can be interpreted as "social distance" or taste divergence.

The evolution of these metrics is presented in \textbf{Figure \ref{fig:anime_metrics}}. As shown in the \textit{upper-left panel},
the average weighted path length rose sharply from 7.1 in 2006 to 44.4 in 2018. This metric represents the "resistance" to navigation:
connecting a fan of a niche genre to a mainstream hit now requires passing through significantly more intermediaries.

Simultaneously, the network diameter (\textit{upper-right panel}) expanded from 29 to 188.5 weighted units. This confirms that the
"taste universe" is expanding. Distinct clusters (e.g., modern idols vs. vintage mecha) are moving mathematically further apart,
creating deep topological fissures.

% --- ОБЪЕДИНЕННАЯ КАРТИНКА (2x2) ---
\begin{figure*}[t]
    \centering
    % Первая строка: Path Length и Diameter
    \begin{subfigure}[b]{0.48\textwidth}
        \centering
        \includegraphics[width=\linewidth]{figures/anime_path_length_evolution.png}
        %\caption{Avg. Weighted Path Length} % Можно убрать подпись, если всё в главной
        \label{fig:path_length}
    \end{subfigure}
    \hfill
    \begin{subfigure}[b]{0.48\textwidth}
        \centering
        \includegraphics[width=\linewidth]{figures/anime_diameter_evolution.png}
        %\caption{Network Diameter}
        \label{fig:diameter}
    \end{subfigure}
    
    \vspace{0.5cm} % Отступ между рядами
    
    % Вторая строка: Clustering и Node Strength
    \begin{subfigure}[b]{0.48\textwidth}
        \centering
        \includegraphics[width=\linewidth]{figures/anime_clustering_coefficient.png}
        %\caption{Avg. Clustering Coefficient}
        \label{fig:clustering}
    \end{subfigure}
    \hfill
    \begin{subfigure}[b]{0.48\textwidth}
        \centering
        \includegraphics[width=\linewidth]{figures/anime_node_strength_2018.png}
        %\caption{Node Strength (2018)}
        \label{fig:node_strength}
    \end{subfigure}
    
    \caption{Evolution of Anime Network Topology (2006–2018). 
    \textbf{Upper-left:} Average Weighted Path Length showing increased navigation difficulty. 
    \textbf{Upper-right:} Network Diameter indicating the expansion of the "taste universe". 
    \textbf{Bottom-left:} Average Clustering Coefficient stabilizing around 0.59, suggesting persistent local cohesion. 
    \textbf{Bottom-right:} Node Strength Distribution (2018) confirming the scale-free ($P(k) \sim k^{-\gamma}$) nature of the modern network.}
    \label{fig:anime_metrics}
\end{figure*}

\subsubsection{Local Cohesion and the "Fandom" Effect}
Despite the global sparsification, the network maintains robust local connectivity. The \textit{bottom-left panel} of
\textbf{Figure \ref{fig:anime_metrics}} illustrates the Average Clustering Coefficient. After an initial adjustment, the metric
stabilized at a remarkably high value of $\approx$0.59. This indicates that the "Small-World" property is preserved locally.
If Anime A is connected to B and C, there is a consistent $\sim$60\% probability that B and C are also connected. This proves that
the sparsification did not destroy community cohesion; instead, the landscape fractured into tight, self-reinforcing "genre bubbles" (fandoms).

\subsubsection{Structural Phase Shift: The 2006 Anomaly}
The year 2006 represents a distinct topological phase. In this nascent period, the network exhibited disassortative mixing
(Degree Assortativity $\approx -0.11$), suggesting a star-like structure where popular titles served as hubs connecting primarily to
niche nodes. From 2007 onward, the network flipped to positive assortativity ($\approx 0.50$), signaling the emergence of the
"Rich-Club" phenomenon.

\subsubsection{Intensity vs. Topology}
Finally, we examine the distribution of influence using Node Strength. The log-log plot in the \textit{bottom-right panel} of
\textbf{Figure \ref{fig:anime_metrics}} confirms the scale-free nature of the modern network, following a clear Power Law distribution.
The network is dominated by a few "mega-hubs," validating the "preferential attachment" growth model.

However, the Assortativity of Strength ($\approx 0.18$) is consistently lower than the Assortativity of Degree ($\approx 0.50$).
This implies that while popular shows are structurally connected, the strongest taste affinities are located in the niche clusters,
not the mainstream core.

% --- USER-USER NETWORK SECTION ---

\subsection{User-User Network}
\label{sec:user_topology}

In stark contrast to the Anime content network—which became "sparse" and harder to traverse as it grew—the User interaction network
exhibits the classic properties of Network Densification. As the community expanded, the social distance between users collapsed,
making the network significantly more interconnected.

While the "universe" of users grew, the social structure did not fragment into isolated islands. Instead, it evolved into a tight,
integrated "global village," where new users actively connected to existing hubs rather than the periphery.

\subsubsection{Global Integration and the "Shrinking World"}
The analysis of weighted path metrics reveals a community that is becoming functionally smaller and easier to traverse, despite growing
in physical size. 

The evolution of these metrics is presented in the \textit{upper panels} of \textbf{Figure \ref{fig:user_metrics}}. As shown in
the \textit{upper-left panel}, the average weighted path length dropped sharply from $\sim$0.45 in 2006 to $\sim$0.31 by 2009,
maintaining this lower baseline through 2018. This reduction is a hallmark of the "Small World" effect: as the platform matured,
users formed bridging connections, accelerating the flow of information across the graph.

Similarly, the network diameter (\textit{upper-right panel}) contracted from $\sim$1.08 to $<0.99$. Unlike the Anime graph, where taste
divergence created massive gaps, the social graph's diameter is shrinking. This indicates that even socially distant groups
(e.g., distinct language communities) are becoming more connected to the mainstream core.

\begin{figure*}[t] % В ACM шаблоне здесь лучше использовать [t] или [b], см. пояснение ниже
    \centering
    % Row 1: Path Length & Diameter
    \begin{subfigure}[b]{0.48\textwidth}
        \centering
        \includegraphics[width=\linewidth]{figures/users_path_length_evolution.png}
        \label{fig:user_path}
    \end{subfigure}
    \hfill
    \begin{subfigure}[b]{0.48\textwidth}
        \centering
        \includegraphics[width=\linewidth]{figures/users_diameter_evolution.png}
        \label{fig:user_diameter}
    \end{subfigure}
    
    \vspace{0.4cm} 
    
    % Row 2: Clustering & Node Strength
    \begin{subfigure}[b]{0.48\textwidth}
        \centering
        \includegraphics[width=\linewidth]{figures/users_clustering_evolution.png}
        \label{fig:user_clustering}
    \end{subfigure}
    \hfill
    \begin{subfigure}[b]{0.48\textwidth}
        \centering
        \includegraphics[width=\linewidth]{figures/users_node_strength_2018.png}
        \label{fig:user_strength}
    \end{subfigure}
    
    \caption{Evolution of User Network Topology (2006–2018). 
    \textbf{Upper-left:} Avg. Weighted Path Length decreasing, indicating higher integration.
    \textbf{Upper-right:} Network Diameter contracting, showing the "shrinking world" phenomenon.
    \textbf{Bottom-left:} Clustering Coefficient peaking in 2009 and slowly stabilizing, reflecting the balance between
    clique formation and expansion.
    \textbf{Bottom-right:} Node Strength Distribution (2018) following a strict Power Law, highlighting the dominance of "Super-Users".}
    \label{fig:user_metrics}
\end{figure*}

\subsubsection{Local Cohesion and Community Structure}
While the network became globally smaller, the local structure evolved to balance rapid growth with intimate social circles.
The \textit{bottom-left panel} of \textbf{Figure \ref{fig:user_metrics}} shows the Average Clustering Coefficient. It peaked
in 2009 ($\sim$0.73) during the platform's initial boom, followed by a gentle decline to $\sim$0.66. 

A score of 0.66 remains exceptionally high for a large social network. The slight decline suggests a natural dilution of cliques as
users added diverse friends, but the high retention proves the community is fundamentally built on strong, overlapping friend groups
rather than loose acquaintances.

\subsubsection{Influence and Inequality}
The distribution of influence confirms a highly stratified social hierarchy. The log-log plot in the \textit{bottom-right panel}
demonstrates a strict linear descent, characteristic of a Scale-Free Network ($P(k) \sim k^{-\gamma}$).

The graph is dominated by a tiny fraction of "Super-Users" (hubs) who possess nearly $1,000\times$ the connectivity of the average user.
These hubs act as the structural "glue" that holds the giant component together, enabling the short path lengths observed in
Section \ref{sec:user_topology}.

% !TeX root = main.tex

\section{Dataset Overview}

MyAnimeList (MAL) is one of the largest online platforms dedicated to anime and manga tracking. 
Users maintain personal lists of watched titles, assign numerical ratings, write reviews, and participate in community discussions. 

Although MAL does not represent the entire global audience, it functions as a \emph{large-scale, organically formed social network} structured around media consumption. Its importance for data analysis stems from several factors:
\begin{itemize}
    \item it has millions of active users with voluntarily provided preference data;
    \item users form clusters and communities around genres, studios, eras, and specific titles;
    \item the rating patterns reflect collective taste dynamics, hype cycles, long-tail phenomena, and cross-cultural differences;
    \item the anime graph (users $\times$ titles) behaves like a sparse bipartite network with hubs, where highly popular titles act as central connectivity points.
\end{itemize}

For these reasons, MAL is a valuable source for studying recommendation systems, preference modeling, popularity prediction, and structural patterns of entertainment consumption.

\subsection{Dataset Structure}

We use the 2020 ``Anime Recommendation Database'' from Kaggle, which combines several processed dumps of MyAnimeList:
\footnote{Where external statistics are used (industry reports, demographic data), and where it is possible, we use sources closest
to 2020 to maintain temporal consistency.}
\paragraph{anime.csv} Metadata for approximately 17,000 titles, with columns including:
\begin{itemize}
    \item identifiers: \texttt{MAL\_ID}, \texttt{Name}, \texttt{English name}, \texttt{Japanese name};
    \item quality and popularity metrics: \texttt{Score}, \texttt{Ranked}, \texttt{Popularity}, \texttt{Members}, \texttt{Favorites};
    \item categorical descriptors: \texttt{Genres}, \texttt{Type}, \texttt{Source}, \texttt{Studios}, \texttt{Producers};
    \item structural info: \texttt{Episodes}, \texttt{Duration}, \texttt{Rating} (age restriction);
    \item engagement counters: \texttt{Watching}, \texttt{Completed}, \texttt{Dropped}, etc.;
    \item per-score vote counts: \texttt{Score-1} $\dots$ \texttt{Score-10}.
\end{itemize}

\paragraph{anime\_with\_synopsis.csv} A reduced version containing \texttt{MAL\_ID}, \texttt{Name}, \texttt{Score}, \texttt{Genres}, \texttt{Synopsis}. Useful for NLP tasks such as clustering by textual description.

\paragraph{animelist.csv} Contains approximately 300 million user–anime interactions, with columns: \texttt{user\_id}, \texttt{anime\_id}, \texttt{rating}, \texttt{watching\_status}, \texttt{watched\_episodes}. This is the core behavioral dataset representing the user–item matrix.

\paragraph{rating\_complete.csv} A filtered version of \texttt{animelist.csv} containing only rows with \texttt{watching\_status = 2} (``Completed''), columns: \texttt{anime\_id}, \texttt{user\_id}, \texttt{rating}. Commonly used for training recommender systems.

\paragraph{watching\_status.csv} Lookup table pairing each integer code with a textual description (``Currently Watching'', ``Completed'', etc.).

\subsection{Strengths of the Dataset}

Despite being collected from an entertainment platform, the dataset has several significant advantages for data science practice:

\begin{itemize}
    \item \textbf{Large scale:} Tens of millions of ratings across thousands of titles enable analysis of long-tail
    distributions, user segmentation, genre-level statistics, and popularity dynamics.

    \item \textbf{Natural heterogeneity of users:} No incentive to game the system; tastes are diverse and clusters form organically.

    \item \textbf{Multiple complementary tables:} Metadata, textual features, and behavioral interactions allow
    content-based, collaborative, graph-based, and hybrid recommender analyses.

    \item \textbf{Excellent for methodological demonstration:} Useful for data cleaning, exploratory data analysis
    (EDA), imputation, outlier detection, recommendation algorithms, and metadata fusion with NLP.

\end{itemize}

\subsection{Limitations and Potential Biases}

The dataset reflects behaviors of a specific community and inherits biases from the platform:

\begin{itemize}
    \item \textbf{Geographical bias:} MAL is most popular in North America, parts of Europe, and Southeast Asia. 
    For example, China is one of the global leaders in anime licensing \cite{animeindustry2020}, but Chinese
    users are absent on MAL.
    
    \item \textbf{Cultural and language bias:} English-speaking communities dominate the dataset.

    \item \textbf{Sparse user--item matrix:} Most users have watched only a small fraction of all anime; 
    this affects collaborative filtering model performance and cold-start dynamics.

    \item \textbf{Artifacts and inconsistency in dataset:} Example for \texttt{rating\_complete.csv}:
    filtering by \texttt{watching\_status = Completed} is not sufficient. Some users marked anime as completed
    but watched fewer episodes than the total and gave non-zero ratings. Since MAL does not allow rating = 0,
    zero ratings indicate \emph{no vote} rather than dislike. These cases are rare (\textasciitilde0.1\%)
    but must be considered during cleaning.

    \item \textbf{No timestamps:} Limits temporal modeling of tastes and popularity.

    \item \textbf{Caution in interpreting popularity:} The dataset reflects preferences of \emph{dedicated fans}
    rather than the general population. So, MAL ratings and engagement metrics do not directly translate to global popularity.
    MAL users are biased towards committed anime fans; casual viewers are underrepresented.

    \item \textbf{Licensing bias:} In many regions, anime consumption occurs primarily through piracy,
    so official licensing or platform metrics severely underestimate actual viewership\footnote{Even in the USA,
    the leader in licensed anime, nearly half of users watch primarily via unofficial services \cite{animeresearch2020}}.

    \item \textbf{MAL - only one of the biggest anime hubs:} Country-level popularity is not reliably inferable
    without additional corrections or external data sources; such adjustments are beyond the scope of this project.

    \end{itemize}
\end{itemize}
% !TeX root = main.tex

\section{Synthetic User Generation}

To simulate global user interactions while preserving privacy, we generate synthetic user profiles based on demographic and traffic data.

\subsection{Data Sources}

\begin{itemize}
    \item \textbf{World cities and populations:} Dataset \cite{citiesdistribution}. Used columns \texttt{ASCII Name, Country Name EN,
    Population, Latitude, Longitude}.

    \item \textbf{MyAnimeList traffic per country:} Scraped manually from semrush.com for October 2025, with columns
    \texttt{Country, Number of Visitors}.

    \item \textbf{Age and sex distributions:} From demographic studies\cite{animeresearch2020} providing mean and standard
    deviation for age, and male/female proportions.
\end{itemize}

Additionally, the demographic survey\cite{animeresearch2020} collected country-level data, but it was conducted at a large anime convention, 
so the sample is a \emph{convenience sample} and not representative of the general population. 
Nevertheless, the observed distributions roughly align with the traffic patterns obtained from semrush.com.

The survey also collected additional information, such as self-reported life satisfaction, hobbies, gender,
preferred decades of favorite titles, and other personal attributes.  
These variables are not directly used in this project, as our analysis relies on synthetic profiles generated
from traffic and demographic distributions.

\subsection{Generation Procedure}

\begin{enumerate}
    \item Compute country-level user proportions based on MAL traffic:

    \[
    P_\text{country} = \frac{\text{MAL users in country X}}{\text{Total MAL users}}.
    \]

    \item Distribute users to cities within each country proportionally to city population:

    \[
    P_\text{city} = P_\text{country} \cdot \frac{\text{City population}}{\sum \text{City populations in country}}.
    \]

    \item Sample user attributes:
    \begin{itemize}
        \item \texttt{age} $\sim$ Normal distribution (mean, std from \cite{animeresearch2020}),
        \item \texttt{sex} $\sim$ Bernoulli(p) (male/female proportion from \cite{animeresearch2020}),
        \item \texttt{latitude, longitude} assigned according to the selected city.
    \end{itemize}

    \item Assign \texttt{user\_id} sequentially and compile all attributes into \texttt{profiles.csv} with columns:

    \[
    \texttt{user\_id, country, city, age, sex, latitude, longitude}.
    \]
\end{enumerate}

\subsection{Notes}

- This synthetic population allows testing recommendation algorithms and demographic analyses without revealing actual user data.
- The proportions reflect December 2025 traffic, but are applied to simulate the 2020 dataset contextually.

% !TeX root = ../main.tex

\section{Dataset and Initial Exploration}

For the analytical part of this project, we use the \textit{Anime Recommendation Database 2020} from Kaggle, which combines metadata about anime titles with large-scale user rating data. The dataset contains over 12{,}000 anime entries and more than 73{,}000 users, resulting in approximately 7.8 million individual rating interactions. This structure naturally splits the data into two main tables:

\begin{itemize}
    \item \textbf{Anime information:} title, genres, type (TV, movie, OVA), number of episodes, release year, studio, popularity metrics, and short descriptions.
    \item \textbf{User ratings:} user identifiers and their numeric ratings for specific anime.
\end{itemize}

At first glance, the data appears rich and diverse, but it also reflects several typical characteristics of real-world datasets:

\begin{itemize}
    \item \textbf{Incomplete records:} many anime lack a release year, genre tags, or episode counts.  
    \item \textbf{Inconsistent categorical data:} genre lists differ in formatting and ordering; some entries use non-standard tags.  
    \item \textbf{Long-tailed distributions:} a small number of highly popular anime dominate ratings, while the majority receive very few.  
    \item \textbf{Sparse user behavior:} most users rate only a tiny fraction of available titles.  
\end{itemize}

These properties are not defects; they represent the typical landscape of large, user-generated media datasets. They also motivate several of the analytical directions in this project, such as identifying rating patterns, studying genre clusters, and modeling the structure of user–anime interactions.

\subsection*{Synthetic Users and Locations}

The original dataset does not contain geographic or demographic information about users. To explore cross-cultural and regional patterns, we generate synthetic user metadata. Each user is assigned a plausible location (country and optionally city), following real-world population distributions. This augmented dataset allows us to ask new types of questions, such as whether certain genres correlate with specific geographic regions, or whether user communities cluster differently across countries.

The synthetic data is clearly separated from the original records and is used only for exploratory purposes, without affecting the underlying rating matrix.

\subsection*{Data Cleaning}

Before we can meaningfully analyze the data, we must resolve the inconsistencies and structural issues inherited from the raw dataset. This involves:

\begin{itemize}
    \item normalizing genre representations and splitting multi-genre fields;
    \item removing or correcting obviously invalid entries (e.g.\ anime with zero episodes released in the 1800s);
    \item handling missing values through imputation or category-specific defaults;
    \item joining anime metadata with synthetic user information to form a unified analytical table;
    \item reducing noise in the user–anime interaction graph by filtering out extremely sparse users or entries.
\end{itemize}

These preprocessing steps create a clean, analyzable foundation for the exploratory data analysis that follows and ensure that all subsequent insights reflect meaningful patterns rather than artifacts of data collection.

\section{Random-Walk Model for User Trajectory Simulation}

\subsection{Motivation}

User activity on a large interaction graph can be interpreted as a sequence of 
transitions between nodes (e.g.\ items, topics, or communities). 
Given such a sequence for each user, our goal is to construct a 
probabilistic model that captures the \emph{structural tendencies} of user navigation.
This model is later used to generate synthetic trajectories---``random walkers''---that
approximate the observed behavior of the real user. 
The ensemble of walkers provides a natural way to measure how typical or atypical 
a given user trajectory is, relative to the structure of the graph.

Because the underlying graph is large (on the order of thousands of nodes and millions 
of edges), all computations must be local and efficient. 
We exploit the fact that the graph evolves year by year, so a user trajectory is 
implicitly aligned with a sequence of yearly graphs.

\subsection{Definition of the Random Walker}

Let $G_t = (V_t, E_t)$ denote the interaction graph in year $t$.  
For a user $u$ we observe a trajectory
\[
    \mathbf{x}^{(u)} = (x_0, x_1, \dots, x_T),
\]
where $x_t \in V_t$ is the node visited by the user in year $t$.
We construct a \emph{random walker} whose behavior in year $t$ is governed only by the 
local structure of $G_t$ and the user's starting point $x_0$.

Formally, for each year $t$ the walker occupies a state $X_t \in V_t$.
Conditioned on $X_t = v$, the walker chooses its position in year $t+1$ according to
a probability distribution over the neighbors of $v$ in $G_t$:
\[
    \mathbb{P}(X_{t+1} = w \mid X_t = v)
    = \frac{1}{\deg_{G_t}(v)}
    \quad \text{for all } (v,w) \in E_t.
\]
That is, the walker performs a uniformly random step along the edges that exist in
the corresponding yearly graph.

A single random walker generates one synthetic trajectory
\[
    \mathbf{Y} = (Y_0, Y_1, \dots, Y_T), \qquad Y_0 = x_0.
\]
To model uncertainty and to obtain stable statistical estimates, we simulate
an ensemble of $K$ independent walkers for each user.

\subsection{Asynchrony and Year-Level Dynamics}

A key detail is that the walkers evolve \emph{asynchronously}.  
Each walker only moves when the global simulation clock advances to a year in which
that walker still has remaining steps.  
This design is necessary because real user trajectories can have different lengths,
and the yearly graphs $G_t$ may differ substantially in size and connectivity.

Thus the simulation proceeds by iterating over years $t = 0,1,\dots,T$
and, for each walker whose trajectory length is greater than $t$, performing
exactly one step in $G_t$.
Walkers whose length is shorter than the current year simply remain inactive.

\subsection{Ensemble-Based Evaluation}

Given a user $u$ with observed trajectory $\mathbf{x}^{(u)}$ and an ensemble of
simulated trajectories $\{\mathbf{Y}^{(k)}\}_{k=1}^K$, we can quantify how 
well the random-walk model explains the user's behavior.

Let $d(\cdot,\cdot)$ be a similarity or distance measure between two trajectories.
In this work we primarily use a weighted node-overlap metric that penalizes 
long-distance mismatches.  
The average similarity of the user to the ensemble is
\[
    \bar{s}^{(u)}
    = \frac{1}{K} \sum_{k=1}^K s\bigl(\mathbf{x}^{(u)}, \mathbf{Y}^{(k)}\bigr),
\]
with an accompanying variance
\[
    \mathrm{Var}^{(u)}
    = \frac{1}{K} \sum_{k=1}^K 
      \Bigl(s\bigl(\mathbf{x}^{(u)}, \mathbf{Y}^{(k)}\bigr) - \bar{s}^{(u)}\Bigr)^2.
\]
These metrics estimate how ``typical'' the user is relative to repeated realizations
of the random-walk model.

Such quantities naturally extend to population-level statistics:
distributions of similarities, identification of outliers, and hypothesis testing
against the null model provided by random walks.

\subsection{Consensus Models}

To summarize the global behavior of the entire ensemble, we consider two forms
of consensus:

\paragraph{Markov Consensus.}
All walker trajectories across all users define empirical transition counts
\[
    C_{vw} = \#\{ \text{times a walker moves } v \rightarrow w \}.
\]
Normalizing the rows yields an empirical transition matrix
\[
    P_{vw}
    = \frac{C_{vw}}{\sum_{w'} C_{vw'}}.
\]
The matrix $P$ defines a global Markov model that captures the average 
transition tendencies dictated by the graph structure and the distribution 
of starting points.
This model can be used to compute likelihoods of real user trajectories,
to generate new synthetic walkers, or to build a deterministic ``most probable''
consensus path by greedy selection.

\paragraph{Medoid Trajectory.}
As a complementary summary, we compute the \emph{medoid} of a set of walker 
trajectories---the trajectory that minimizes the total distance to all others:
\[
    k^\ast
    = \arg\min_{k} \sum_{j=1}^K
      d\bigl(\mathbf{Y}^{(k)}, \mathbf{Y}^{(j)}\bigr).
\]
The medoid offers an interpretable representative path that arises from an
actual random walker, as opposed to the probabilistic object represented by $P$.

\subsection{Interpretation}

The random-walk construction provides an explicit null model driven solely by 
the graph's local connectivity.  
If a real user's trajectory significantly deviates from the ensemble predicted by the graph,
the deviation may reveal hidden structure, atypical behavior, or external influences
that are not captured by topology alone.

Conversely, if the random-walk ensemble closely matches the user, the graph alone
is sufficient to explain the observed behavior.

This duality---graph-driven randomness versus user-specific structure---is the
central object of analysis in the subsequent sections of the report.

% !TeX root = main.tex

\section{MARS\_1.0 Project Tree}

\begin{verbatim}
MARS_1.0/
├── data/
│   ├── anime_ranks/
│   │   ├── anime.csv
│   │   ├── animelist.csv
│   │   ├── anime_with_synopsis.csv
│   │   ├── rating_complete.csv
│   │   └── watching_status.csv
│   ├── anime_timestamps/
│   │   └── anime_timestamps.csv
│   ├── cities_population_and_location/
│   │   └── cities_population_and_location.csv
│   ├── myanimelist_countries_distribution/
│   │   └── myanimelist_countries_distribution.csv
│   └── users/
│       └── profiles.csv
├── db_tools/
│   ├── ...
│   └── ...
├── fds_tools/
│   ├── __init__.py
│   ├── data_cleaner.py
│   ├── fake_user_generator.py (class FakeUsersGenerator)
│   ├── fds_main.py
│   └── project_latex/
│       ├── main.tex
│       ├── chapters/
│       │   ├── introduction.tex
│       │   ├── overview.tex
│       │   ├── fake_user_generation.tex
│       │   ├── dataset_curriculum.tex
│       │   ...
│       │   └── references.bib        
│       └── out/
│           ├── main.pdf
│           └── ...
├── cda_tools/
│   ├── ...
│   └── ...
├── __init__.py
├── .env
├── .gitignore
├── common_tools.py (classes CommonTools, PandasTools, DBTools)
├── datasets.json
├── LICENSE
├── README.md
└── requirements.txt
\end{verbatim}
\bibliographystyle{plain}
\bibliography{chapters/references}
\end{document}
