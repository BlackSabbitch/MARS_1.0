\section{Predictive Modeling of User Migration}

In this chapter\footnote{The complete experimental pipeline and visualization tools are available in the project
repository: \texttt{project\_cda/3\_user\_migration\_pipeline.ipynb}.}, we address the problem of user migration between communities,
a fundamental challenge in dynamic social network analysis. We model the MAL user network as an evolving graph $G_t = (V_t, E_t)$,
where nodes represent unique users and weighted edges denote social proximity based on shared anime consumption history. Since the
statistical and topological properties of the network fluctuate significantly over time, our analysis focuses on transitions between
consecutive temporal snapshots, denoted as $t$ and $t+1$.

The migration prediction task is formalized as a binary classification problem. Let $C(u, t)$ denote the community assignment
of user $u$ at time $t$. A user is labeled as a \textit{migrant} if their community membership changes between snapshots:

\begin{equation}
    y_u = \begin{cases} 
    1 & \text{if } C(u, t+1) \neq C(u, t) \\
    0 & \text{if } C(u, t+1) = C(u, t) 
    \end{cases}
\end{equation}

Features extracted at time $t$ are utilized to predict the state $y_u$ at time $t+1$.

\subsection{Motivation for Community Detection Algorithm Choice}
To identify distinct user communities within each temporal snapshot, we employ the \textbf{Leiden algorithm}. While sharing the objective
of modularity maximization with the classical Louvain method, Leiden is selected for its superior stability, scalability, and topological
guarantees on large-scale networks.

The primary advantage of Leiden over Louvain lies in its handling of community connectivity, scalability and reliability on large networks.
While Louvain is efficient, it can produce poorly connected or even disconnected communities, Leiden addresses this through a distinct
\textit{refinement phase}, guaranteeing that all resulting clusters are well-connected. This is critical for our study, as membership in
a poorly connected cluster could represent an algorithmic artifact rather than genuine social cohesion.


Alternative community detection methods were evaluated but rejected due to specific limitations:
\begin{itemize}
    \item \textbf{Edge-betweenness (e.g., Girvan–Newman):} Computationally prohibitive with a complexity of $O(|V||E|^2)$, rendering
    it impractical for our network ($|V| \approx 50k$, $|E| \approx 20M$).
    \item \textbf{Infomap:} While efficient, it is sensitive to edge weight distributions and tends to generate highly fragmented
    micro-communities in dense graphs.
    \item \textbf{Label Propagation:} Although extremely fast, its non-deterministic nature leads to unstable partitions, making
    longitudinal tracking unreliable.
\end{itemize}

Leiden combines efficiency, stability, and high-quality partitions. Its near-linear scalability allows us to handle the network effectively,
while the refinement phase ensures structurally meaningful communities.

Leiden is a hierarchical algorithm, producing a dendrogram of partitions controlled by the resolution parameter $\gamma$.
\begin{itemize}
    \item Lower values ($\gamma < 1$) favor larger, macro-communities (e.g., broad genre preferences).
    \item Higher values ($\gamma > 1$) reveal finer micro-communities (e.g., specific niche groups).
\end{itemize}

In this study, we fixed $\gamma = 1$. This choice provides a balanced granularity, capturing meaningful community structures without
merging distinct groups or isolating noise-driven micro-clusters. It also establishes a robust baseline for migration analysis.

To bridge the gap between topology and semantics, we constructed a correlation matrix between the detected user communities and anime
genre clusters based on historical rating data. The resulting heatmap, complemented by alluvial flow diagrams, reveals which anime
categories dominate within each user group. This approach allows us to uncover community-specific preference profiles and explore the
content-driven factors underlying user migration.

\subsection{Feature Engineering}
To capture the factors driving migration, we constructed a multidimensional feature space describing the user's state at time $t$. The
feature vector $X_u^{(t)}$ comprises four distinct categories:

\begin{enumerate}
    \item \textbf{Graph Structure (Global \& Local):} Measures of centrality and influence, including Degree, Weighted Strength, PageRank,
    and K-core decomposition.
    \item \textbf{Local Cohesion:} This is captured by the weighted clustering coefficient; low values indicate weak neighborhood integration
    and a potentially higher migration risk.
    \item \textbf{Community Embeddedness:} Metrics quantifying the boundary position of a user. Key among these is the
    \textit{Intra-Community Ratio (ICR)}, defined as the fraction of a user's edges that connect to nodes within the same community.
    \item \textbf{Temporal Dynamics:} Delta features representing the year-over-year change in structural metrics
    (e.g., $\Delta \text{Degree}$, $\Delta \text{ICR}$), capturing the trajectory of a user's engagement.
    \item \textbf{Demographics \& Activity:} Static attributes (gender, age, location) integrated with dynamic indicators, such as
    the number of watched titles and rating fluctuations over time.
\end{enumerate}

This results in a comprehensive per-user, per-year feature vector integrating static, structural, and dynamic signals for predictive modeling.

\subsection{Model Specification and Performance Assessment}
We utilized \textbf{CatBoost}, a gradient boosting algorithm on decision trees, for the classification task. CatBoost was selected for its
native handling of categorical features (e.g., community ID, location) without one-hot encoding, robustness to missing values and different
feature scales, and ability to model complex non-linear interactions between structural and behavioral signals. Crucially, it offers native
support for class imbalance via internal weighting, allowing the algorithm to effectively handle the rarity of migration events.

Considering the task complexity and strong class imbalance (migration events constitute $\approx 17\%$ of the data), the trained model
demonstrates strong predictive performance. It achieves a \textbf{ROC AUC of 0.915}, indicating strong discrimination capability.

\begin{table}[h]
\centering
\caption{Classification Performance Metrics}
\label{tab:model_performance}
\begin{tabular}{l c c c}
\hline
\textbf{Class} & \textbf{Precision} & \textbf{Recall} & \textbf{F1-Score} \\
\hline
Non-Migrant (0) & 0.93 & 0.92 & 0.93 \\
Migrant (1) & 0.64 & 0.68 & 0.66 \\
\hline
\textbf{Overall Accuracy} & \multicolumn{3}{c}{0.88} \\
\hline
\end{tabular}
\end{table}

These results (Table \ref{tab:model_performance}) confirm that user migration is not a stochastic process but a predictable phenomenon driven
by observable network and behavioral characteristics. This validates the model's utility for early-warning detection, offering threshold
adjustments to balance recall and precision.

\subsection{Feature Importance and Behavioral Interpretation}
To explain the model's decisions, we analyzed SHAP (SHapley Additive exPlanations) summary plot. Figure \ref{fig:shap_summary} illustrates
the global feature importance and directional impact on the prediction.

\begin{figure}[h]
    \centering
    \includegraphics[width=0.9\linewidth]{figures/shap_summary.png}
    \caption{Distribution of SHAP values indicating the contribution of each feature to the model’s predictions.}
    \label{fig:shap_summary}
\end{figure}

The analysis reveals that \textbf{Intra-Community Ratio (ICR)} is the dominant predictor. A high ICR acts as a stabilizing force,
anchoring the user within their group, whereas a low ICR serves as a strong precursor to migration. Notably, the
\textit{temporal decrease} in ICR is the most informative early-warning dynamic signal.

The \textbf{current community assignment} itself also has a strong effect, indicating that baseline migration risks differ significantly across
communities (i.e., some communities are inherently more unstable or "leaky" than others).

\textbf{PageRank} exhibits a non-linear stabilizing effect: highly influential users are less likely to migrate, suggesting that social capital
creates "inertia." Demographically, older users show lower migration propensities.

These results emphasize that user migration cannot be captured by singular metrics. Only a comprehensive analysis combining network topology,
temporal changes, and behavioral features reveals the true mechanisms driving community shifts.