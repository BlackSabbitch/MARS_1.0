\section{Random-Walk Model for User Trajectory Simulation}

\subsection{Motivation}

User activity on a large interaction graph can be interpreted as a sequence of 
transitions between nodes (e.g.\ items, topics, or communities). 
Given such a sequence for each user, our goal is to construct a 
probabilistic model that captures the \emph{structural tendencies} of user navigation.
This model is later used to generate synthetic trajectories---``random walkers''---that
approximate the observed behavior of the real user. 
The ensemble of walkers provides a natural way to measure how typical or atypical 
a given user trajectory is, relative to the structure of the graph.

Because the underlying graph is large (on the order of thousands of nodes and millions 
of edges), all computations must be local and efficient. 
We exploit the fact that the graph evolves year by year, so a user trajectory is 
implicitly aligned with a sequence of yearly graphs.

\subsection{Definition of the Random Walker}

Let $G_t = (V_t, E_t)$ denote the interaction graph in year $t$.  
For a user $u$ we observe a trajectory
\[
    \mathbf{x}^{(u)} = (x_0, x_1, \dots, x_T),
\]
where $x_t \in V_t$ is the node visited by the user in year $t$.
We construct a \emph{random walker} whose behavior in year $t$ is governed only by the 
local structure of $G_t$ and the user's starting point $x_0$.

Formally, for each year $t$ the walker occupies a state $X_t \in V_t$.
Conditioned on $X_t = v$, the walker chooses its position in year $t+1$ according to
a probability distribution over the neighbors of $v$ in $G_t$:
\[
    \mathbb{P}(X_{t+1} = w \mid X_t = v)
    = \frac{1}{\deg_{G_t}(v)}
    \quad \text{for all } (v,w) \in E_t.
\]
That is, the walker performs a uniformly random step along the edges that exist in
the corresponding yearly graph.

A single random walker generates one synthetic trajectory
\[
    \mathbf{Y} = (Y_0, Y_1, \dots, Y_T), \qquad Y_0 = x_0.
\]
To model uncertainty and to obtain stable statistical estimates, we simulate
an ensemble of $K$ independent walkers for each user.

\subsection{Asynchrony and Year-Level Dynamics}

A key detail is that the walkers evolve \emph{asynchronously}.  
Each walker only moves when the global simulation clock advances to a year in which
that walker still has remaining steps.  
This design is necessary because real user trajectories can have different lengths,
and the yearly graphs $G_t$ may differ substantially in size and connectivity.

Thus the simulation proceeds by iterating over years $t = 0,1,\dots,T$
and, for each walker whose trajectory length is greater than $t$, performing
exactly one step in $G_t$.
Walkers whose length is shorter than the current year simply remain inactive.

\subsection{Ensemble-Based Evaluation}

Given a user $u$ with observed trajectory $\mathbf{x}^{(u)}$ and an ensemble of
simulated trajectories $\{\mathbf{Y}^{(k)}\}_{k=1}^K$, we can quantify how 
well the random-walk model explains the user's behavior.

Let $d(\cdot,\cdot)$ be a similarity or distance measure between two trajectories.
In this work we primarily use a weighted node-overlap metric that penalizes 
long-distance mismatches.  
The average similarity of the user to the ensemble is
\[
    \bar{s}^{(u)}
    = \frac{1}{K} \sum_{k=1}^K s\bigl(\mathbf{x}^{(u)}, \mathbf{Y}^{(k)}\bigr),
\]
with an accompanying variance
\[
    \mathrm{Var}^{(u)}
    = \frac{1}{K} \sum_{k=1}^K 
      \Bigl(s\bigl(\mathbf{x}^{(u)}, \mathbf{Y}^{(k)}\bigr) - \bar{s}^{(u)}\Bigr)^2.
\]
These metrics estimate how ``typical'' the user is relative to repeated realizations
of the random-walk model.

Such quantities naturally extend to population-level statistics:
distributions of similarities, identification of outliers, and hypothesis testing
against the null model provided by random walks.

\subsection{Consensus Models}

To summarize the global behavior of the entire ensemble, we consider two forms
of consensus:

\paragraph{Markov Consensus.}
All walker trajectories across all users define empirical transition counts
\[
    C_{vw} = \#\{ \text{times a walker moves } v \rightarrow w \}.
\]
Normalizing the rows yields an empirical transition matrix
\[
    P_{vw}
    = \frac{C_{vw}}{\sum_{w'} C_{vw'}}.
\]
The matrix $P$ defines a global Markov model that captures the average 
transition tendencies dictated by the graph structure and the distribution 
of starting points.
This model can be used to compute likelihoods of real user trajectories,
to generate new synthetic walkers, or to build a deterministic ``most probable''
consensus path by greedy selection.

\paragraph{Medoid Trajectory.}
As a complementary summary, we compute the \emph{medoid} of a set of walker 
trajectories---the trajectory that minimizes the total distance to all others:
\[
    k^\ast
    = \arg\min_{k} \sum_{j=1}^K
      d\bigl(\mathbf{Y}^{(k)}, \mathbf{Y}^{(j)}\bigr).
\]
The medoid offers an interpretable representative path that arises from an
actual random walker, as opposed to the probabilistic object represented by $P$.

\subsection{Interpretation}

The random-walk construction provides an explicit null model driven solely by 
the graph's local connectivity.  
If a real user's trajectory significantly deviates from the ensemble predicted by the graph,
the deviation may reveal hidden structure, atypical behavior, or external influences
that are not captured by topology alone.

Conversely, if the random-walk ensemble closely matches the user, the graph alone
is sufficient to explain the observed behavior.

This duality---graph-driven randomness versus user-specific structure---is the
central object of analysis in the subsequent sections of the report.
