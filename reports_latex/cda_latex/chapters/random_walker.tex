\section{Stochastic Simulation of User Navigation}
\label{sec:random_walker}

In this chapter\footnote{The complete experimental pipeline and visualization tools are available in the project
repository: \texttt{project\_cda/3\_random\_walker.ipynb}.}, we investigate whether user migration patterns can be explained by purely topological
factors. To this end, we introduce a stochastic agent—the \textbf{Random Walker} — which serves as a null model for user navigation.
By simulating thousands of probabilistic trajectories on the evolving graph, we test the "Structural Determinism" hypothesis: that
a user's future community is primarily determined by the connectivity of their current position.

\subsection{Formal Definition of the Topological Agent}

Let $G_t = (V_t, E_t)$ denote the state of the network at year $t$. We model the user's journey as a sequence of community memberships:
$$ \mathcal{T}_u = (c_0, c_1, \dots, c_T) $$
where $c_t$ is the community ID of user $u$ at time $t$.

The Random Walker agent operates under the assumption of Markovian neutrality. Conditioned on being in node $v$ at time $t$,
the probability of transitioning to node $w$ at $t+1$ is proportional to the edge weight $w_{vw}$:
\begin{equation}
    P(X_{t+1} = w \mid X_t = v) = \frac{w_{vw}}{\sum_{k \in \mathcal{N}(v)} w_{vk}}
\end{equation}
This formulation implies that the agent has no memory and no specific preferences other than the structural "path of least resistance"
offered by the graph topology.

\subsection{Simulation Protocol}

To generate a statistical baseline, we execute the following protocol for every user in the dataset:
\begin{enumerate}
    \item \textbf{Initialization:} An ensemble of $K=100$ stochastic agents is spawned at the user's actual position in year $t=2006$.
    \item \textbf{Propagation:} For each subsequent year $t+1$, agents transition to a new community based on the transition matrix of
    the snapshot $G_t$.
    \item \textbf{Consensus:} The resulting cloud of $K$ trajectories forms a probabilistic distribution of possible futures. We define
    the \textit{Predicted Community} $\hat{c}_{t+1}$ as the mode of this distribution (Majority Vote).
\end{enumerate}

This approach effectively turns the Random Walker into a classifier: if the user's actual next step coincides with the walker's most probable
step, we consider the behavior "structurally predictable."

\subsection{Divergence from Empirical Trajectories}

Comparing the simulated trajectories with empirical user data reveals a fundamental disconnect between topological probability and human choice.
We quantified the similarity between the Random Walker's predicted path and the user's actual migration history using a population-wide
similarity metric (Jaccard overlap of visited communities).

The results indicate a near-total divergence:
\begin{itemize}
    \item \textbf{Mean Similarity:} $0.0435$ ($\pm 0.0184$)
    \item \textbf{Median Similarity:} $0.0447$
\end{itemize}

These extremely low values ($\approx 4.3\%$) imply that user navigation in the MAL ecosystem is **not stochastic**. While the network structure
defines the \textbf{possibilities} (the set of accessible neighbors), it does not dictate the *probabilities*. The "gravity" of large structural
hubs, which dominates the Random Walker's decisions, fails to explain the nuanced, content-driven choices made by real users.

\subsection{Conclusion: The Need for Supervised Learning}
The failure of the Random Walker experiment demonstrates that the MAL ecosystem is not merely a system of passive diffusion. Users are not
particles flowing down topological gradients; their migration is driven by active choices, likely influenced by specific content features,
social signals, and external trends not captured by edge weights alone.

This finding provides the motivation for the next chapter. Since simple Markovian dynamics are insufficient, we must turn to
\textbf{Supervised Machine Learning} (Section \ref{sec:ml_migration}) to capture the non-linear interactions between structural metrics and user
behavior.