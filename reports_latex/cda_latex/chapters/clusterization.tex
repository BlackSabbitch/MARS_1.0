% !TeX root = main.tex

\section{Mesoscale Analysis and Community Detection Experiments}
\label{sec:experiments}


Having characterized\footnote{The complete experimental pipeline and visualization tools are available in the project
repository: \texttt{project\_cda/2\_clusterization.ipynb}.} the global topological evolution, we shift
our focus to the mesoscale level—identifying the distinct taste communities that constitute the anime ecosystem. This section details
the experimental framework used to select the optimal
clustering algorithm, ensuring that the detected communities are structurally robust, semantically meaningful, and suitable for the subsequent
migration analysis.

\subsection{Experimental Setup}
For each annual snapshot $G_t$ of the projected Anime-Anime network (2006–2018), we conducted a comparative analysis of five primary community
detection algorithms:
\begin{enumerate}
    \item \textbf{Leiden (Modularity):} Tested with resolution parameters $\gamma \in [0.5, 2.0]$.
    \item \textbf{Infomap (Map Equation):} Tested with Markov time parameters $T \in \{10, 25, 50\}$.
    \item \textbf{Leading Eigenvector:} A spectral method based on modularity maximization.
    \item \textbf{Label Propagation (LPA):} Used as a baseline heuristic.
\end{enumerate}

The quality of partitions was evaluated using a composite set of metrics, prioritizing \textbf{Modularity ($Q$)} for structural definition,
\textbf{Genre Purity} for semantic alignment, and \textbf{Number of Clusters ($N_{cl}$)} for interpretability.

\subsection{Comparative Analysis: The Structure-Granularity Trade-off}
The experiments revealed a clear trade-off between semantic precision and structural compactness (see Table \ref{tab:clustering_results}).

\textbf{Label Propagation (LPA)} proved to be an outlier with suboptimal performance. It demonstrated the lowest Modularity
($Q \approx 0.03$), failing to detect the dense community structure known to exist in the network.

\textbf{Leading Eigenvector} showed extreme instability. While its average modularity ($\approx 0.30$) was acceptable, the number of
clusters fluctuated wildly across years (ranging from 4 to 108), indicating high sensitivity to minor topological changes. Such volatility
makes it unsuitable for longitudinal tracking.

\textbf{Infomap} exhibited behavior highly dependent on the Markov time parameter $T$:
\begin{itemize}
    \item At $T=10$ ("short walks"), it resulted in \textbf{hyper-fragmentation}, producing over 200 micro-communities with high
    semantic purity ($\sim 0.63$) but poor modular structure ($Q \approx 0.19$).
    \item At $T=50$, it achieved the highest semantic accuracy overall (Genre Purity $\sim 0.61$) while maintaining decent
    modularity ($Q \approx 0.35$). However, it still divided the network into $\approx 29$ clusters, which is too granular for
    macro-level migration analysis.
\end{itemize}

\textbf{Leiden (Modularity)} demonstrated superior structural definition. It consistently achieved the highest Modularity scores
($Q \approx 0.36$). Crucially, at $\gamma=1.0$, it struck a "Golden Mean":
\begin{itemize}
    \item \textbf{High Modularity:} $0.358$ (comparable to the best results).
    \item \textbf{Acceptable Purity:} $0.57$ (close to Infomap's 0.61).
    \item \textbf{Optimal Compactness:} It identified $\approx 10$ stable macro-communities.
\end{itemize}

% TABLE START
\begin{table*}[t]
\centering
\caption{Average Performance of Clustering Algorithms (2006–2018)}
\label{tab:clustering_results}
\begin{tabular}{l c c c c l}
\hline
\textbf{Algorithm} & \textbf{Modularity} & \textbf{Genre Purity} & \textbf{Source Purity} & \textbf{N Clusters} & \textbf{Verdict} \\
\hline
Label Propagation & 0.033 & 0.31 & 0.35 & 6 & Underfitting \\
Leading Eigenvector & 0.301 & 0.50 & 0.45 & 34 & \textbf{Unstable} \\
Infomap ($T=10$) & 0.192 & \textbf{0.63} & \textbf{0.54} & $\sim 150$ & Hyper-segmentation \\
Infomap ($T=50$) & 0.345 & 0.61 & \textbf{0.54} & 29 & Over-segmentation \\
Leiden ($\gamma=0.9$) & 0.357 & 0.00\textsuperscript{*} & 0.00\textsuperscript{*} & 6 & Semantic Mismatch \\
\textbf{Leiden ($\gamma=1.0$)} & \textbf{0.358} & 0.57 & 0.53 & \textbf{10} & \textbf{Selected} \\
\hline
\multicolumn{6}{l}{\footnotesize \textsuperscript{*} Values of 0.00 indicate data corruption for specific years due to tag misalignment
during evaluation.}
\end{tabular}
\end{table*}
% TABLE END

\subsection{Selection Logic}
Based on these results, we prioritized structural interpretability over maximal semantic purity. For the task of predicting user migration,
it is more valuable to track transitions between a manageable number of distinct "Macro-Communities" (e.g., "Mainstream Action" vs.
"Vintage Mecha") rather than tracking noise across 100+ fragmented micro-clusters (as seen in Infomap T=10 or Eigenvector). Therefore,
we selected \textbf{Leiden ($\gamma=1.0$)} as the primary algorithm.

\subsection{Resolution Parameter Sweep}
Having selected the Leiden algorithm, we performed a detailed sensitivity analysis to confirm the stability of the chosen
resolution $\gamma=1.0$. We constructed heatmaps for Modularity and Cluster Counts across all years for $\gamma \in [0.1, 3.0]$.

\begin{figure}[b]
    \centering
    % ПЕРВЫЙ ГРАФИК: МОДУЛЯРНОСТЬ
    \begin{subfigure}[b]{0.48\textwidth}
        \centering
        \includegraphics[width=\linewidth]{figures/leiden_res_mod.png}
        \caption{Evolution of Modularity ($Q$).}
        \label{fig:heatmap_mod}
    \end{subfigure}
    \hfill
    % ВТОРОЙ ГРАФИК: КОЛИЧЕСТВО КЛАСТЕРОВ
    \begin{subfigure}[b]{0.48\textwidth}
        \centering
        \includegraphics[width=\linewidth]{figures/leiden_res_clust.png}
        \caption{Evolution of Cluster Count ($N$).}
        \label{fig:heatmap_ncl}
    \end{subfigure}
    
    \caption{Multi-resolution analysis of the Leiden algorithm (2006–2018). \textbf{(a)} Modularity remains robust across the optimal range.
    \textbf{(b)} The number of clusters exhibits a stable "plateau" around $\gamma=1.0$ ($\sim 10$ communities).}
    \label{fig:resolution_sweep}
\end{figure}

As illustrated in \textbf{Figure \ref{fig:resolution_sweep}}, the combined analysis confirms our choice:
\begin{itemize}
    \item \textbf{Modularity Landscape:} We observe high modularity values persisting across the range $\gamma \in [0.8, 1.2]$, suggesting
    that the strong community structure is not an artifact of a specific parameter point.
    \item \textbf{Cluster Count Plateau:} Simultaneously, the number of clusters stabilizes in this range. Unlike higher resolutions
    ($\gamma > 1.5$), where the network splinters, $\gamma=1.0$ consistently yields $\sim 10$ distinct communities.
\end{itemize}

This topological stability provides a solid foundation for the longitudinal tracking of user migration.