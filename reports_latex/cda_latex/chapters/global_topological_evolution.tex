% !TeX root = main.tex

\section{Topological Evolution of Projected Networks}

In this section, we analyze the structural evolution of the projections derived from the bipartite graph. We first examine the Anime-Anime
network to understand how content relationships shifted over time, followed by an analysis of the User-User network.

% --- ПОДРАЗДЕЛ 3.1: АНИМЕ ---
\subsection{Anime-Anime Network}
\label{sec:anime_topology}

The projected anime-anime network experienced explosive growth over the analyzed period (2006–2018), transforming from a compact,
niche community into a sprawling, heterogeneous ecosystem. This transformation is defined by three primary phenomena: the
densification-sparsification paradox, increasing taste divergence, and the crystallization of a "rich-club" core.

\subsubsection{The Densification-Sparsification Paradox}
The network underwent a dramatic scale expansion: the number of nodes (anime titles) increased from 732 in 2006 to 6,129 in 2018,
while the volume of connections (edges) surged from $\sim$64,000 to $\sim$819,000. However, this volumetric growth reveals a fundamental
structural shift.

While the absolute number of connections increased by an order of magnitude, the potential number of connections grew quadratically ($N^2$).
Consequently, the global Graph Density declined precipitously from 0.2387 (2006) to 0.0436 (2018).

This indicates a transition from a "Village" topology—where the community is small enough for high interconnectedness—to a "Metropolis"
structure. In the modern era, the ecosystem has become highly specialized; while the total volume of interactions is higher, individual
anime titles connect to a significantly smaller fraction of the total population. The network has shifted from a monolithic block to a
spread-out, sparse landscape.

\subsubsection{Increasing Social Distance and Taste Divergence}
To quantify the "cost" of traversing this expanding network, we analyzed weighted path metrics. Since edge weights represent similarity
(Jaccard), the weighted distance can be interpreted as "social distance" or taste divergence.

The evolution of these metrics is presented in \textbf{Figure \ref{fig:anime_metrics}}. As shown in the \textit{upper-left panel},
the average weighted path length rose sharply from 7.1 in 2006 to 44.4 in 2018. This metric represents the "resistance" to navigation:
connecting a fan of a niche genre to a mainstream hit now requires passing through significantly more intermediaries.

Simultaneously, the network diameter (\textit{upper-right panel}) expanded from 29 to 188.5 weighted units. This confirms that the
"taste universe" is expanding. Distinct clusters (e.g., modern idols vs. vintage mecha) are moving mathematically further apart,
creating deep topological fissures.

% --- ОБЪЕДИНЕННАЯ КАРТИНКА (2x2) ---
\begin{figure*}[t]
    \centering
    % Первая строка: Path Length и Diameter
    \begin{subfigure}[b]{0.48\textwidth}
        \centering
        \includegraphics[width=\linewidth]{figures/anime_path_length_evolution.png}
        %\caption{Avg. Weighted Path Length} % Можно убрать подпись, если всё в главной
        \label{fig:path_length}
    \end{subfigure}
    \hfill
    \begin{subfigure}[b]{0.48\textwidth}
        \centering
        \includegraphics[width=\linewidth]{figures/anime_diameter_evolution.png}
        %\caption{Network Diameter}
        \label{fig:diameter}
    \end{subfigure}
    
    \vspace{0.5cm} % Отступ между рядами
    
    % Вторая строка: Clustering и Node Strength
    \begin{subfigure}[b]{0.48\textwidth}
        \centering
        \includegraphics[width=\linewidth]{figures/anime_clustering_coefficient.png}
        %\caption{Avg. Clustering Coefficient}
        \label{fig:clustering}
    \end{subfigure}
    \hfill
    \begin{subfigure}[b]{0.48\textwidth}
        \centering
        \includegraphics[width=\linewidth]{figures/anime_node_strength_2018.png}
        %\caption{Node Strength (2018)}
        \label{fig:node_strength}
    \end{subfigure}
    
    \caption{Evolution of Anime Network Topology (2006–2018). 
    \textbf{Upper-left:} Average Weighted Path Length showing increased navigation difficulty. 
    \textbf{Upper-right:} Network Diameter indicating the expansion of the "taste universe". 
    \textbf{Bottom-left:} Average Clustering Coefficient stabilizing around 0.59, suggesting persistent local cohesion. 
    \textbf{Bottom-right:} Node Strength Distribution (2018) confirming the scale-free ($P(k) \sim k^{-\gamma}$) nature of the modern network.}
    \label{fig:anime_metrics}
\end{figure*}

\subsubsection{Local Cohesion and the "Fandom" Effect}
Despite the global sparsification, the network maintains robust local connectivity. The \textit{bottom-left panel} of
\textbf{Figure \ref{fig:anime_metrics}} illustrates the Average Clustering Coefficient. After an initial adjustment, the metric
stabilized at a remarkably high value of $\approx$0.59. This indicates that the "Small-World" property is preserved locally.
If Anime A is connected to B and C, there is a consistent $\sim$60\% probability that B and C are also connected. This proves that
the sparsification did not destroy community cohesion; instead, the landscape fractured into tight, self-reinforcing "genre bubbles" (fandoms).

\subsubsection{Structural Phase Shift: The 2006 Anomaly}
The year 2006 represents a distinct topological phase. In this nascent period, the network exhibited disassortative mixing
(Degree Assortativity $\approx -0.11$), suggesting a star-like structure where popular titles served as hubs connecting primarily to
niche nodes. From 2007 onward, the network flipped to positive assortativity ($\approx 0.50$), signaling the emergence of the
"Rich-Club" phenomenon.

\subsubsection{Intensity vs. Topology}
Finally, we examine the distribution of influence using Node Strength. The log-log plot in the \textit{bottom-right panel} of
\textbf{Figure \ref{fig:anime_metrics}} confirms the scale-free nature of the modern network, following a clear Power Law distribution.
The network is dominated by a few "mega-hubs," validating the "preferential attachment" growth model.

However, the Assortativity of Strength ($\approx 0.18$) is consistently lower than the Assortativity of Degree ($\approx 0.50$).
This implies that while popular shows are structurally connected, the strongest taste affinities are located in the niche clusters,
not the mainstream core.

% --- USER-USER NETWORK SECTION ---

\subsection{User-User Network}
\label{sec:user_topology}

In stark contrast to the Anime content network—which became "sparse" and harder to traverse as it grew—the User interaction network
exhibits the classic properties of Network Densification. As the community expanded, the social distance between users collapsed,
making the network significantly more interconnected.

While the "universe" of users grew, the social structure did not fragment into isolated islands. Instead, it evolved into a tight,
integrated "global village," where new users actively connected to existing hubs rather than the periphery.

\subsubsection{Global Integration and the "Shrinking World"}
The analysis of weighted path metrics reveals a community that is becoming functionally smaller and easier to traverse, despite growing
in physical size. 

The evolution of these metrics is presented in the \textit{upper panels} of \textbf{Figure \ref{fig:user_metrics}}. As shown in
the \textit{upper-left panel}, the average weighted path length dropped sharply from $\sim$0.45 in 2006 to $\sim$0.31 by 2009,
maintaining this lower baseline through 2018. This reduction is a hallmark of the "Small World" effect: as the platform matured,
users formed bridging connections, accelerating the flow of information across the graph.

Similarly, the network diameter (\textit{upper-right panel}) contracted from $\sim$1.08 to $<0.99$. Unlike the Anime graph, where taste
divergence created massive gaps, the social graph's diameter is shrinking. This indicates that even socially distant groups
(e.g., distinct language communities) are becoming more connected to the mainstream core.

\begin{figure*}[t] % В ACM шаблоне здесь лучше использовать [t] или [b], см. пояснение ниже
    \centering
    % Row 1: Path Length & Diameter
    \begin{subfigure}[b]{0.48\textwidth}
        \centering
        \includegraphics[width=\linewidth]{figures/users_path_length_evolution.png}
        \label{fig:user_path}
    \end{subfigure}
    \hfill
    \begin{subfigure}[b]{0.48\textwidth}
        \centering
        \includegraphics[width=\linewidth]{figures/users_diameter_evolution.png}
        \label{fig:user_diameter}
    \end{subfigure}
    
    \vspace{0.4cm} 
    
    % Row 2: Clustering & Node Strength
    \begin{subfigure}[b]{0.48\textwidth}
        \centering
        \includegraphics[width=\linewidth]{figures/users_clustering_evolution.png}
        \label{fig:user_clustering}
    \end{subfigure}
    \hfill
    \begin{subfigure}[b]{0.48\textwidth}
        \centering
        \includegraphics[width=\linewidth]{figures/users_node_strength_2018.png}
        \label{fig:user_strength}
    \end{subfigure}
    
    \caption{Evolution of User Network Topology (2006–2018). 
    \textbf{Upper-left:} Avg. Weighted Path Length decreasing, indicating higher integration.
    \textbf{Upper-right:} Network Diameter contracting, showing the "shrinking world" phenomenon.
    \textbf{Bottom-left:} Clustering Coefficient peaking in 2009 and slowly stabilizing, reflecting the balance between
    clique formation and expansion.
    \textbf{Bottom-right:} Node Strength Distribution (2018) following a strict Power Law, highlighting the dominance of "Super-Users".}
    \label{fig:user_metrics}
\end{figure*}

\subsubsection{Local Cohesion and Community Structure}
While the network became globally smaller, the local structure evolved to balance rapid growth with intimate social circles.
The \textit{bottom-left panel} of \textbf{Figure \ref{fig:user_metrics}} shows the Average Clustering Coefficient. It peaked
in 2009 ($\sim$0.73) during the platform's initial boom, followed by a gentle decline to $\sim$0.66. 

A score of 0.66 remains exceptionally high for a large social network. The slight decline suggests a natural dilution of cliques as
users added diverse friends, but the high retention proves the community is fundamentally built on strong, overlapping friend groups
rather than loose acquaintances.

\subsubsection{Influence and Inequality}
The distribution of influence confirms a highly stratified social hierarchy. The log-log plot in the \textit{bottom-right panel}
demonstrates a strict linear descent, characteristic of a Scale-Free Network ($P(k) \sim k^{-\gamma}$).

The graph is dominated by a tiny fraction of "Super-Users" (hubs) who possess nearly $1,000\times$ the connectivity of the average user.
These hubs act as the structural "glue" that holds the giant component together, enabling the short path lengths observed in
Section \ref{sec:user_topology}.